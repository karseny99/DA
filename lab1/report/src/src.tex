\section{Описание}
Поразрядная сортировка подразумевает применение сортировки подсчетом по каждому разряду элемента последовательности. 

Для оценки сложности поразрядной сортировки вспомним сложность сортировки подсчетом: $\Theta(n + k)$, где n - количество элементов последовательности, а k - максимум этой последовательности. 
Поскольку мы запускаем сортировку подсчетом для каждого знака ключа, то получаем, что оценка поразрядной сортировки равна $O(n * l)$, где n - количество элементов, а l - количество разрядов. 


\pagebreak

\section{Исходный код}
На каждой непустой строке входного файла располагается пара \enquote{ключ-значение}, поэтому создадим новую 
структуру {\ttfamily TValue}, в которой будем хранить ключ в виде статического массива  {\ttfamily char[]} и значение в виде числа типа {\ttfamily int64\_t}. 

\begin{lstlisting}[language=C]
#include <string>

const int array_digit_size = 16;
const int key_length = 32;

class TValue {   
public:
    char key[key_length];
    uint64_t value; 

    TValue() = default;
    ~TValue() = default;
    
    TValue(std::string& _key, uint64_t _value) {
        for(int i = 0; i < key_length; ++i) 
            key[i] = _key[i];

        value = _value;
    }
    
    void set(std::string& _key, uint64_t _value) {

        for(int i = 0; i < key_length; ++i) 
            key[i] = _key[i];

        value = _value;
    }
};

\end{lstlisting}

В файле \textbf{sort.cpp} находится реализация поразрядной сортировки. \\
Здесь в цикле происходит запуск сортировки подсчетом для каждого знака ключа. По окончанию сортировки подсчетом для i-го разряда происходит обмен буферами вспомогательного и результирующего векторов.

\begin{lstlisting}[language=C]
#include "sort.hpp"
#include <iostream>


namespace sort {

    const int array_digit_size = 16;
    const int key_length = 32;

    void radix_sort( vector::Vector<TValue>& elems ) {

        vector::Vector<TValue> tmpResult(elems.get_size() + 1, true);

        for(int j = key_length - 1; j >= 0; --j) {

            int tmp[16];
            for(int k = 0; k < array_digit_size; ++k) 
                tmp[k] = 0;
            

            for(int i = 0; i < elems.get_size(); ++i) {
                if('0' <= elems[i].key[j] and elems[i].key[j] <= '9')
                    ++tmp[elems[i].key[j] - '0'];
                else 
                    ++tmp[elems[i].key[j] - 'a' + 10];
            }

            for(int k = 1; k < 16; ++k) {
                tmp[k] += tmp[k - 1];
            }

            for(int i = elems.get_size() - 1; i >= 0; --i) {
                int key;
                if('0' <= elems[i].key[j] and elems[i].key[j] <= '9')
                    key = elems[i].key[j] - '0';
                else    
                    key = elems[i].key[j] - 'a' + 10;

                int pos = tmp[key]--;
                tmpResult[pos-1] = elems[i];
            }
            vector::swap(elems, tmpResult);
        }

        tmpResult.~Vector();
    }
}
\end{lstlisting}

\pagebreak

\section{Консоль}
\begin{alltt}
karseny99@karseny99:/mnt/study/DA/lab1$ make
g++ -std=c++20  -c sort.cpp
g++ -std=c++20  sort.o lab1.cpp -o lab1
karseny99@karseny99:/mnt/study/DA/lab1$ ./lab1 < tests/02.t
11b3a318b750add4482f67160fd1c7af        5506620685087310468
2904a441415ad22c0ccbc8bab780ff01        15631025563904442920
2d57ead15d45f3b95217fc26f64e624f        17967741058834373087
5fb9cc898b5dc6ca6478b0f61ca58579        15549767237243378484
7d685d1f631b298beb848e4845e7d70b        13093843595709548303
88f32327a698e79035ea91b25c9fe79f        1522945456683991812
9ebaa12c5c594e3c6cf597c586120fc3        1741633272260555417
c23fe20a6b98698e44b378628cb02624        14690737309650550198
f2ad2b0ee93e3e0f8c94dac8b8181bf1        10823747237399279286
f7b994993cd43e8a1e8bc567008a5c34        3420891951338888443
\end{alltt}
\pagebreak
